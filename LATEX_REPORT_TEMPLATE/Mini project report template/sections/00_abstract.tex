% Abstract page
\section*{Abstract}
This project, Doom64 Online, revisits the classic Doom 64 game, introducing a LAN-based multiplayer aspect. Central to the project is the development of an efficient server-client architecture to ensure smooth data exchange among multiple players. The game offers a first-person shooter experience with continuous gameplay until server shutdown. Key features include robust multi-threaded server support, seamless join-and-play mechanics, and client-side rendering for fluid graphics. Player interactions, like movement and shooting, are managed through server-client communications using JSON for data transfer. Resources for the graphical user interface were adapted from an existing repository, focusing mainly on the computer networking components of the game. This project serves as a model for implementing multiplayer functionality in retro-styled games, leveraging modern network programming techniques.
\vspace{.5cm}

\textbf{Keywords:} Multiplayer, LAN, FPS, Doom 64, Server-Client Architecture, Networking, Game Development, Python, Pygame, JSON
\newpage